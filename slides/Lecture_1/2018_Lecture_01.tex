%% LyX 2.0.7.1 created this file.  For more info, see http://www.lyx.org/.
%% Do not edit unless you really know what you are doing.
\documentclass[english]{beamer}
\usepackage[T1]{fontenc}
\usepackage[latin9]{inputenc}
\usepackage{listings}
\setcounter{secnumdepth}{3}
\setcounter{tocdepth}{3}
\setlength{\parindent}{0bp}
\usepackage{babel}
\ifx\hypersetup\undefined
  \AtBeginDocument{%
    \hypersetup{unicode=true}
  }
\else
  \hypersetup{unicode=true}
\fi

\makeatletter
%%%%%%%%%%%%%%%%%%%%%%%%%%%%%% Textclass specific LaTeX commands.
 % this default might be overridden by plain title style
 \newcommand\makebeamertitle{\frame{\maketitle}}%
 \AtBeginDocument{
   \let\origtableofcontents=\tableofcontents
   \def\tableofcontents{\@ifnextchar[{\origtableofcontents}{\gobbletableofcontents}}
   \def\gobbletableofcontents#1{\origtableofcontents}
 }
 \long\def\lyxframe#1{\@lyxframe#1\@lyxframestop}%
 \def\@lyxframe{\@ifnextchar<{\@@lyxframe}{\@@lyxframe<*>}}%
 \def\@@lyxframe<#1>{\@ifnextchar[{\@@@lyxframe<#1>}{\@@@lyxframe<#1>[]}}
 \def\@@@lyxframe<#1>[{\@ifnextchar<{\@@@@@lyxframe<#1>[}{\@@@@lyxframe<#1>[<*>][}}
 \def\@@@@@lyxframe<#1>[#2]{\@ifnextchar[{\@@@@lyxframe<#1>[#2]}{\@@@@lyxframe<#1>[#2][]}}
 \long\def\@@@@lyxframe<#1>[#2][#3]#4\@lyxframestop#5\lyxframeend{%
   \frame<#1>[#2][#3]{\frametitle{#4}#5}}
 \def\lyxframeend{} % In case there is a superfluous frame end

%%%%%%%%%%%%%%%%%%%%%%%%%%%%%% User specified LaTeX commands.
\usetheme[secheader]{Madrid}

\makeatletter
\setbeamertemplate{footline}
{
  \leavevmode%
  \hbox{%
  \begin{beamercolorbox}[wd=.333333\paperwidth,ht=2.25ex,dp=1ex,center]{author in head/foot}%
    \usebeamerfont{author in head/foot}{Methods and Computing}
  \end{beamercolorbox}%
  \begin{beamercolorbox}[wd=.333333\paperwidth,ht=2.25ex,dp=1ex,center]{title in head/foot}%
    \usebeamerfont{title in head/foot}{Harvard University}
  \end{beamercolorbox}%
  \begin{beamercolorbox}[wd=.333333\paperwidth,ht=2.25ex,dp=1ex,right]{date in head/foot}%
    \usebeamerfont{date in head/foot}{Department of Biostatistics}
    \insertframenumber{} / \inserttotalframenumber\hspace*{2ex} 
  \end{beamercolorbox}}%
  \vskip0pt%
}
\makeatother

\makeatother

\newtheorem{mydef}{Definition}[section]

%% Update the formatting of \alert{}
\setbeamerfont{alerted text}{series=\bfseries}
\setbeamercolor{alerted text}{fg=black}

\begin{document}

\title{Biostatistics Preparatory Course:\\
Methods and Computing}


\author{Lecture 1}


\date{Intro, R Basics, Data Types/Structures}

\makebeamertitle

\lyxframeend{}\lyxframe{Introductions}
\begin{itemize}
\item Instructor: Izzie Fulcher
\item Email: isabelfulcher@g.harvard.edu
\end{itemize}


\lyxframeend{}


\lyxframeend{}\lyxframe{Structure of the Course}

\begin{itemize}
\item Focus on \emph{basic} programming in R
\item Statistical topics will be mixed in
\pause
\item Each class will consist of:

\begin{itemize}
\item A short lecture on topics in R relevant for biostatistics (20-30 minutes)
\item For the remainder of the session, we will complete several exercises, often broken up into groups
\end{itemize}
\item Solutions to exercises and slides will be posted after the each session
\item Bringing a laptop is recommended, since programming is most
  easily learned by \emph{doing}
\end{itemize}

\lyxframeend{}

\lyxframeend{}\lyxframe{Topics covered}

The first two sessions will go over basics in R programming, then we will start to mix in statistical topics:
\begin{itemize}
		\item Probability distributions  
		\item Linear regression (if time, generalized linear models)
		\item Monte Carlo simulations
		\item Maximum likelihood estimation
		\item Bootstrap
\end{itemize}

\pause

All programming will be done in R, but we will also use,
\begin{itemize}
	\item Rmarkdown and LaTeX
	\item Cluster computing 
\end{itemize}


\lyxframeend{}


\lyxframeend{}\lyxframe{What is R?}
\begin{itemize}
\item Free programming language for statistical computing and graphics
\item Easy way to distribute one's own packages for novel methods
\end{itemize}

\lyxframeend{}


\lyxframeend{}\lyxframe{How to get R and Rstudio}
\begin{itemize}
\item R --- the language and GUI (Graphical User Interface)

\begin{itemize}
\item \href{https://cran.r-project.org}{https://cran.r-project.org}
\end{itemize}
\pause
\item Rstudio --- an IDE (Integrated Development Environment)

\begin{itemize}
\item \href{http://www.rstudio.com/products/rstudio/download/}{http://www.rstudio.com/products/rstudio/download/}
\end{itemize}
\end{itemize}

\vspace{3mm}

\lyxframeend{}

\begin{frame}[fragile]{R Basics}
% \begin{definition}[Comment]
%   Any line of text preceded by a ``\#''
% \end{definition}
% \begin{itemize}
% \item Example:\\ \verb+# This is a comment+
% \end{itemize}

\begin{definition}[Variables]
  Names that are assigned values (which can be of various data types)
  \begin{lstlisting}[basicstyle={\footnotesize\ttfamily},language=R,showstringspaces=false]
# Example
a <- 3
\end{lstlisting}
\end{definition}

\begin{itemize}
%\item Do not need declaration in R
\item Some variables are built in
  (eg., \verb+pi+)
%\item Be careful with name choices---can override built-in names
\end{itemize}
\pause
\begin{definition}[Functions]
A piece of code that performs a task (may take one or more
arguments)\\

\begin{lstlisting}[basicstyle={\footnotesize\ttfamily},language=R,showstringspaces=false]
# Examples
log(3)
exp(3)
\end{lstlisting}
\end{definition}

\begin{itemize}
\item Multiple arguments in functions are separated by commas
\end{itemize}
\end{frame}



\begin{frame}[fragile]{R Basics: Types of Commands}
\begin{definition}[Assignment]
The command is evaluated, its value passed to variable and not
printed\\

\begin{lstlisting}[basicstyle={\footnotesize\ttfamily},language=R,showstringspaces=false]
# Examples:
a <- a + 3
b <- log(3)
\end{lstlisting}
\end{definition}
\pause
\begin{definition}[Expression]
The command is evaluated, its value is printed and not retained\\
 
\begin{lstlisting}[basicstyle={\footnotesize\ttfamily},language=R,showstringspaces=false]
# Examples:
1 + 1
a + 3
log(a)
\end{lstlisting}
\end{definition}

\end{frame}

\begin{frame}[fragile]{Removing objects from workspace}
\begin{itemize}
	\item Objects stored in the workspace can be displayed with \verb+ls()+ 
	\item Variables can be deleted with  \verb+rm()+ 
	\item  \verb+rm(list = ls())+ will remove all objects from the current workspace
\end{itemize}

\end{frame}

\begin{frame}[fragile]{R basics: Packages}
\begin{itemize}
\item Packages are ways to distribute content and may contain:

\begin{itemize}
\item Datasets
\item Functions
\end{itemize}
\item Packages must first be installed and loaded before they can
be used:


\noindent 
\begin{lstlisting}[basicstyle={\footnotesize\ttfamily},language=R,showstringspaces=false]
# Example:
install.packages("matrixStats")
library("matrixStats")
\end{lstlisting}

\item You can check the documentation for a package with \verb+help(package = `packagename')+ 

\end{itemize}
\end{frame}





\begin{frame}[fragile]{Exercise 1}
\begin{itemize}
\item Try out commands of your own, both assignment and expression:
\begin{enumerate}
\item Create a variable $x$ with value 2
\item Subtract 7 from $x$
\item Create a new variable $y  = x - 7$
\end{enumerate}
\item Install and load a package
  \begin{itemize}
  \item \verb+ggplot2+
  \item \verb+mvtnorm+
  \item \verb+stringr+
  \end{itemize}
\item Use basic functions, such as: \\
 \verb+log()+, \verb+exp()+, \verb+abs()+, \verb+sqrt()+, \verb+ceiling()+, \verb+trunc()+
%\item Familiarize yourself with Rstudio
\item Practice using the help (? followed by function name)
  \begin{itemize}
  \item \verb+ceiling+
  \end{itemize}
\item Once you have completed the above, remove all objects from your workspace
\end{itemize}

\end{frame}





\begin{frame}[fragile]{Common Data Types in R}
\begin{definition}[Character]
Always in quotations and do not have numerical value
\vspace{-0.05in}
\begin{lstlisting}[basicstyle={\footnotesize\ttfamily},language=R,showstringspaces=false]
# Example:
"abc123"
\end{lstlisting}
\vspace{-0.1in}
\end{definition}
\pause
\begin{definition}[Boolean]
A logical value that takes one of the values TRUE or FALSE
\vspace{-0.05in}
\begin{lstlisting}[basicstyle={\footnotesize\ttfamily},language=R,showstringspaces=false]
# Example:
TRUE
\end{lstlisting}
\vspace{-0.1in}
\end{definition}
\pause
\begin{definition}[Numeric]
A numeric value (integer or decimal)
\vspace{-0.05in}
\begin{lstlisting}[basicstyle={\footnotesize\ttfamily},language=R,showstringspaces=false]
# Examples:
24
pi
\end{lstlisting}
\vspace{-0.1in}
\end{definition}
\end{frame}

\begin{frame}[fragile]{Common Data Structures in R: Vectors}
\begin{itemize}
\item Ways to create vectors:


\noindent 
\begin{lstlisting}[basicstyle={\footnotesize\ttfamily},language=R,showstringspaces=false]
vec1 <- c(1,2,3)
vec2 <- 1:10
vec3 <- seq(1,10,0.5)
vec4 <- rep(1,10)
\end{lstlisting}

\pause

\item Vectors are referenced by the index for both assignment and expression:


\noindent 
\begin{lstlisting}[basicstyle={\footnotesize\ttfamily},language=R,showstringspaces=false]
vec1[1]
vec1[1] <- 6
\end{lstlisting}


\end{itemize}
\end{frame}

\begin{frame}[fragile]{Common Data Structures in R: Matrices}
\begin{itemize}
\item Can create a matrix from scratch:
  \noindent
  \begin{lstlisting}[basicstyle={\footnotesize\ttfamily}, language=R,
    showstringspaces=false]
mymat <- matrix(1:10, nrow=2, ncol=5, byrow=TRUE) 
mymat2 <- matrix(0, nrow=3,ncol=4)
  \end{lstlisting}
  \begin{itemize}
  \item If the first argument to \verb+matrix+ is a vector, 
    \verb+nrow*ncol+ must be the length of the vector
  \item If the first argument is a scalar, the matrix will contain
    only the scalar value
  \end{itemize}


\item Can also combine vectors of same length to form matrices:


\noindent 
\begin{lstlisting}[basicstyle={\footnotesize\ttfamily},language=R,showstringspaces=false]
vec1 <- c(1, 2, 3)
vec2 <- c(2, 3, 4)
mat1 <- rbind(vec1, vec2)
mat2 <- cbind(vec1, vec2)
\end{lstlisting}

\pause

\item Matrices are referenced by their row and column number for assignment
and expression:


\noindent 
\begin{lstlisting}[basicstyle={\footnotesize\ttfamily},language=R,showstringspaces=false]
mymat[1, 2]
mymat[1, 2] <- 6
\end{lstlisting}


\end{itemize}
\end{frame}

\begin{frame}[fragile]{Other Data Structures in R: Combining Data Types}
\begin{itemize}
\item Vectors and matrices contain one type of data at a time
% \item There are other more complex data structures in R that can accommodate
% multiple data types
\end{itemize}
\begin{definition}[List]
A collection of objects of multiple types
\vspace{-0.05in}
\begin{lstlisting}[basicstyle={\footnotesize\ttfamily},language=R,showstringspaces=false]
a <- 1
b <- "a string :)"
mylist <- list(a, b)
mylist[[1]]
\end{lstlisting}
\vspace{-0.1in}
\end{definition}
\pause
\begin{definition}[Data frame]
A list of vectors of the same length (can be manipulated like a
matrix)
\vspace{-0.05in}
\begin{lstlisting}[basicstyle={\footnotesize\ttfamily},language=R,showstringspaces=false]
x <- c("a", "b", "c")
y <- c(TRUE, TRUE, FALSE)
z <- data.frame(x, y)
z[1, 1]
z$x
\end{lstlisting}
\vspace{-0.1in}
\end{definition}
\end{frame}

\begin{frame}[fragile]{Some Useful Commands}
\begin{itemize}
	\item \verb+class()+ will display the object class
	\item\verb+dim()+ will display the dimension of a matrix
	\item \verb+length()+ will display the number of elements in an object
\end{itemize}
\end{frame}

\begin{frame}[fragile]{Exercise 2}
\begin{itemize}
\item Create a $3 \times 4$ matrix, $M$, from a vector of length 12 containing all 1s
\item Create 3 vectors of length 5: one character, one numeric and one boolean called $char.vec$, $num.vec$, and $bool.vec$ respectively
\item Create a data frame called $mydata$ with the 3 vectors you created
\item Create a list called $mylist$ with the 3 vectors you created
\item Convert $M$ to a data frame. Check you have done this using \verb+class()+
\item Name the columns of $M$ using \verb+colnames()+ 
\end{itemize}
\end{frame}

\begin{frame}[fragile]{Generating empty structures}
\begin{itemize}
	\item Often, it's useful to make empty ``containers'' into which you
	will put results
	\item To make empty lists or vectors: \verb+vector+ function
	\begin{lstlisting}[basicstyle={\footnotesize\ttfamily}, language=R,
	showstringspaces=false]
	empty.list     <- vector(length=3, mode="list")
	empty.num.vec  <- vector(length=2, mode="numeric")
	empty.char.vec <- vector(length=5, mode="character")
	\end{lstlisting}
	\item To make an empty matrix: make a matrix filled with \verb+NA+
	\begin{lstlisting}[basicstyle={\footnotesize\ttfamily}, language=R,
	showstringspaces=false]
	empty.matrix <- matrix(NA, nrow=5, ncol=6)
	\end{lstlisting}
	\begin{itemize}
		\item This makes it easy to see if something didn't get filled in
	\end{itemize}
\end{itemize}
\end{frame}


\end{document}







\lyxframeend{}\lyxframe{Rstudio}
\includegraphics<1>[width=\textwidth, page=1, clip=true, trim=0in 0.5in
0in 0.5in]{RStudio_window.pdf}
\includegraphics<2>[width=\textwidth, page=2, clip=true, trim=0in 0.5in
0in 0.5in]{RStudio_window.pdf}
\includegraphics<3>[width=\textwidth, page=3, clip=true, trim=0in 0.5in
0in 0.5in]{RStudio_window.pdf}
\includegraphics<4>[width=\textwidth, page=4, clip=true, trim=0in 0.5in
0in 0.5in]{RStudio_window.pdf}
\includegraphics<5>[width=\textwidth, page=5, clip=true, trim=0in 0.5in
0in 0.5in]{RStudio_window.pdf}
\lyxframeend{}

t
